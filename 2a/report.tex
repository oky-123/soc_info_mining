% platex report && dvipdfmx report.dvi && open report.pdf
\documentclass{jsarticle}
\usepackage{listings,jlisting}
\usepackage[dvipdfmx]{graphicx}
\lstset{
  basicstyle={\ttfamily},
  identifierstyle={\small},
  commentstyle={\smallitshape},
  keywordstyle={\small\bfseries},
  ndkeywordstyle={\small},
  stringstyle={\small\ttfamily},
  frame={tb},
  breaklines=true,
  columns=[l]{fullflexible},
  numbers=left,
  xrightmargin=0zw,
  xleftmargin=3zw,
  numberstyle={\scriptsize},
  stepnumber=1,
  numbersep=1zw,
  lineskip=-0.5ex
}

\begin{document}

\title{情報システム論実習 演習課題 5月19日分}
\author{6930318812 沖野 雄哉}
\maketitle

\section{モンテカルロシミュレーション}

\subsection{コード}
\begin{lstlisting}[caption=MonteCarlo.R]
M <- 100
estimated_pi_values = 1:M

for(i in 1:M) {
    N <- 9999
    x <- runif(N, 0, 1)
    y <- runif(N, 0, 1)
    sum <- x^2 + y^2

    z = length(sum[sum<1])
    v <- z * 4 / N
    estimated_pi_values[i] <- v
}

message(mean(estimated_pi_values))
\end{lstlisting}

\subsection{出力結果}
3.14139813981398

\subsection{結果}
$\pi$の値3.1415$\ldots$と近い値が得られた。
モンテカルロ・シミュレーションの結果はどのような分布となるか
100回行って調べてみると以下の図1のような結果となった。
平均が$\pi$の正規分布のような分布となることが確認された。

\newpage

\begin{figure}[h]
    \centering
    \caption[モンテカルロ法による$\pi$の予測値の分布]{モンテカルロ法による$\pi$の予測値の分布}
    \includegraphics[width=14cm]{EstimatedPiValues.png}
\end{figure}

\newpage

\section{標本平均の平均}

\subsection{コード}

\begin{lstlisting}[caption=Norm.R]
sample_means = 1:100
for(i in 1:100) {
  x <- rnorm(100, 20, sqrt(9))
  sample_means[i] <- mean(x)
}
## pngで保存
png("SampleMeans.png",width=12,height=12,res=300,unit="cm")
hist(sample_means, main="Sample means", xlab="mean")

## 平均, 分散の比較
message("Mean: ", mean(sample_means))
message("Variance: ", var(sample_means))
\end{lstlisting}

\subsection{出力結果}
Mean: 19.9889456645273\\
Variance: 0.0927933611128982\\

また、標本平均の分布は以下図2のようになった。

\begin{figure}[h]
    \centering
    \caption[標本平均の分散]{標本平均の分散}
    \includegraphics[width=14cm]{SampleMeans.png}
\end{figure}

\subsection{考察}
結果から、平均は20、分散は9/100に近い値が得られた。
また、ヒストグラムから
標本平均の分布は平均$\mu$、分散$\sigma^2/n$の正規分布に従うことが確認された。
\end{document}
